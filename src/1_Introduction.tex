\chapter{Tổng quan về đề tài}
\section{Giới thiệu}
\tab Trong thời đại công nghệ số phát triển mạnh mẽ, việc phát triển giao diện người dùng (GUI) cho các hệ thống nhúng đang trở thành một nhu cầu thiết yếu trong các ứng dụng công nghiệp, dân dụng và IoT. Giao diện trực quan không chỉ giúp người dùng tương tác thuận tiện hơn với hệ thống, mà còn nâng cao giá trị sử dụng và tính chuyên nghiệp của sản phẩm.\\ 
\tab Với sự phổ biến của vi điều khiển ESP32-S3 và sự phát triển của các thư viện mã nguồn mở như LVGL (Light and Versatile Graphics Library), việc xây dựng các giao diện đồ họa phức tạp trên các hệ thống nhúng trở nên khả thi và hiệu quả. LVGL cung cấp đầy đủ các thành phần UI như nút bấm, slider, bảng thông tin... và hỗ trợ cảm ứng, animation, đa ngôn ngữ, rất phù hợp cho các ứng dụng có màn hình tích hợp.\\ 
\tab Trong khuôn khổ đề tài \textbf{“Phát triển thư viện giao diện cho mạch nhúng ESP32 sử dụng thư viện đồ họa LVGL”}, nhóm chúng em tập trung nghiên cứu, thiết kế và hiện thực một thư viện giao diện đồ họa tùy biến, sử dụng ESP32-S3 kết hợp với màn hình cảm ứng 7 inch. Giao diện này sẽ hỗ trợ hiển thị và tương tác với các thông tin cảm biến như nhiệt độ, độ ẩm và trạng thái các thiết bị điều khiển trong mô hình thử nghiệm.Dự án cũng tích hợp kết nối với Web Server riêng, cho phép ESP32 đồng bộ dữ liệu cảm biến và nhận lệnh điều khiển thông qua giao thức HTTP hoặc WebSocket. Việc này hỗ trợ khả năng giám sát và điều khiển từ xa qua giao diện web một cách linh hoạt, không phụ thuộc vào nền tảng IoT bên thứ ba.
\section{Mục tiêu}
\tab Mục tiêu chính của đề tài là xây dựng một thư viện giao diện người dùng trực quan dành cho hệ thống nhúng sử dụng vi điều khiển ESP32-S3 và thư viện LVGL. Thư viện này sẽ giúp đơn giản hóa quá trình phát triển ứng dụng trên nền tảng ESP32 có tích hợp màn hình cảm ứng. Các mục tiêu cụ thể bao gồm:

\begin{itemize} 
\item \textbf{Tìm hiểu và tích hợp LVGL vào hệ thống ESP32-S3}: Cấu hình hệ thống phần mềm để chạy thư viện LVGL mượt mà trên màn hình cảm ứng 7 inch sử dụng giao tiếp SPI/Parallel. 
\item \textbf{Thiết kế thư viện giao diện tái sử dụng}: Phát triển các thành phần UI có thể tái sử dụng như các nút điều khiển, thẻ thông tin cảm biến, màn hình chính, màn hình cài đặt, v.v... phục vụ cho các hệ thống nhúng có màn hình. 
\item \textbf{Tích hợp cảm biến và hệ thống điều khiển}: Hiển thị thông tin từ các cảm biến nhiệt độ, độ ẩm (kết nối RS485 hoặc UART) và điều khiển thiết bị đầu ra (qua relay) thông qua giao diện đồ họa. 
\item \textbf{Kết nối và tương tác với Web Server}: ESP32 có khả năng gửi dữ liệu cảm biến đến Web Server thông qua HTTP hoặc WebSocket, đồng thời nhận lệnh điều khiển để cập nhật trạng thái giao diện và các thiết bị đầu ra. Hệ thống có thể được giám sát và điều khiển từ xa qua trình duyệt web.
\item \textbf{Tối ưu hiệu năng}: Đảm bảo giao diện hoạt động mượt mà, tiêu tốn tài nguyên hợp lý và có khả năng mở rộng cho các ứng dụng thực tế khác. 
\end{itemize}
\section{Thực trạng}
\tab Việc phát triển giao diện người dùng trên các thiết bị nhúng trước đây thường gặp nhiều khó khăn do hạn chế về tài nguyên phần cứng, thiếu công cụ hỗ trợ và chi phí cao. Tuy nhiên, trong những năm gần đây, sự xuất hiện của các thư viện đồ họa nhẹ như LVGL đã thay đổi đáng kể cục diện.\\ 
\tab Mặc dù LVGL đã được cộng đồng mã nguồn mở sử dụng rộng rãi, nhưng việc tích hợp và sử dụng thư viện này trên ESP32-S3 vẫn đòi hỏi kiến thức chuyên sâu về cấu trúc hệ thống nhúng, quản lý bộ nhớ, cảm ứng và hiển thị. Ngoài ra, hầu hết các giao diện hiện tại vẫn chưa tối ưu về tính trực quan, khả năng tái sử dụng và mở rộng.\\ 
\tab Do đó, việc phát triển một thư viện giao diện chuyên dụng, dễ tùy chỉnh, thân thiện với người dùng và phù hợp với ứng dụng IoT thực tế là điều cần thiết.

\section{Phạm vi dự án}
\tab Phạm vi nghiên cứu của đề tài tập trung vào việc phát triển và thử nghiệm thư viện giao diện đồ họa trên hệ thống phần cứng bao gồm: \begin{itemize} 
\item ESP32-S3 kết hợp với màn hình cảm ứng 7 inch. 
\item Giao diện được xây dựng bằng LVGL, hiển thị dữ liệu cảm biến (nhiệt độ, độ ẩm). 
\item Hỗ trợ điều khiển thiết bị đầu ra thông qua các nút điều khiển trên giao diện. 
\item Tích hợp truyền nhận dữ liệu cảm biến qua giao tiếp RS485. 
\item Giao tiếp với Web Server nội bộ hoặc trên mạng thông qua giao thức HTTP/WebSocket để truyền nhận dữ liệu và thực hiện điều khiển từ xa.
\end{itemize} 
\tab Đề tài không đi sâu vào phát triển phần cứng cấp thấp hay các thuật toán điều khiển phức tạp, mà tập trung vào phần giao diện, tính năng kết nối và khả năng mở rộng.

\section{Ý nghĩa thực tiễn}  
\tab Việc phát triển thư viện giao diện sử dụng LVGL trên ESP32-S3 không chỉ giúp chuẩn hóa và tối ưu quy trình thiết kế giao diện cho hệ thống nhúng, mà còn mang lại nhiều ý nghĩa thực tiễn như: 
\begin{enumerate} 
\item \textbf{Tăng tính tương tác người dùng}: Cung cấp giao diện trực quan, hiện đại cho các thiết bị IoT hoặc hệ thống nhúng, giúp người dùng dễ thao tác và theo dõi. 
\item \textbf{Rút ngắn thời gian phát triển sản phẩm}: Thư viện giao diện có thể tái sử dụng, giúp đẩy nhanh quá trình tích hợp và triển khai các sản phẩm nhúng mới. 
\item \textbf{Tăng cường tính chuyên nghiệp}: Giao diện đẹp và ổn định góp phần nâng cao giá trị của sản phẩm công nghệ và tăng tính cạnh tranh. 
\item \textbf{Khả năng mở rộng cao}: Thư viện có thể được áp dụng cho nhiều loại thiết bị khác nhau như máy đo môi trường, bảng điều khiển, hệ thống giám sát từ xa, v.v. 
\end{enumerate}

\tab Đề tài góp phần vào việc xây dựng nền tảng phần mềm mở cho các hệ thống nhúng hiện đại, đặc biệt là trong lĩnh vực nhà thông minh, nông nghiệp công nghệ cao và công nghiệp 4.0.


\newpage

