\begin{preface}{Lời cam đoan}
    \tab Chúng em xin cam đoan rằng báo cáo luận văn với đề tài "Phát triển thư viện giao diện cho mạch nhúng ESP32 sử dụng thư viện đồ họa LVGL" là kết quả của quá trình nghiên cứu và thực hiện nghiêm túc của nhóm, dưới sự hướng dẫn tận tình của thầy Lê Trọng Nhân. Toàn bộ nội dung, số liệu và kết quả trong báo cáo đều do nhóm tự triển khai, không sao chép từ bất kỳ nguồn tài liệu hay công trình nào khác. Mọi sự hỗ trợ và các tài liệu tham khảo đều được trích dẫn rõ ràng và minh bạch. Nhóm chúng em xin hoàn toàn chịu trách nhiệm nếu có bất kỳ vi phạm nào liên quan đến tính trung thực và bản quyền trong báo cáo này.
    \begin{flushright}
    Thành phố Hồ Chí Minh, tháng 4 năm 2025.
    \end{flushright}
    \end{preface}
    
    \newpage
    
    \begin{preface}{Lời cảm ơn}
    \tab Chúng em xin bày tỏ lòng biết ơn chân thành đến thầy Lê Trọng Nhân, người đã tận tình hướng dẫn và đồng hành cùng nhóm trong suốt quá trình thực hiện đề tài. Thầy luôn theo sát tiến độ, góp ý kịp thời và tạo mọi điều kiện thuận lợi để nhóm có thể hoàn thành tốt nhiệm vụ được giao.\\
    \tab Đồng thời, những ý kiến đóng góp từ thầy không chỉ giúp chúng em hoàn thiện nội dung báo cáo mà còn là nguồn động viên lớn trong suốt quá trình học tập và nghiên cứu.\\
    \tab Chúng em xin kính chúc thầy luôn mạnh khỏe, thành công trong sự nghiệp giảng dạy và tiếp tục truyền cảm hứng học tập, nghiên cứu cho các thế hệ sinh viên tiếp theo.\\
    \tab Mặc dù đã nỗ lực để hoàn thành đề tài trong phạm vi thời gian và kiến thức hiện có, nhưng chắc chắn vẫn còn những thiếu sót. Chúng em rất mong nhận được thêm ý kiến đóng góp từ quý thầy cô để hoàn thiện hơn trong tương lai.
    \begin{flushright}
    Chúng em xin chân thành cảm ơn.\\
    Thành phố Hồ Chí Minh, tháng 4 năm 2025.
    \end{flushright}
    \end{preface}
    
    \newpage
    
    \begin{preface}{Tóm tắt đồ án}
    \tab Đề tài tập trung vào việc phát triển một thư viện giao diện đồ họa cho mạch nhúng ESP32, sử dụng thư viện mã nguồn mở LVGL (Light and Versatile Graphics Library). Mục tiêu là xây dựng một hệ thống giao diện người dùng trực quan, linh hoạt, tối ưu cho các ứng dụng nhúng trong điều kiện tài nguyên hạn chế. Thư viện được phát triển nhằm phục vụ cho việc thiết kế, lập trình và hiển thị các thành phần giao diện như nút nhấn, biểu đồ, thông báo trạng thái… trên các màn hình cảm ứng sử dụng với vi điều khiển ESP32.\\
    \tab Trong quá trình thực hiện, nhóm đã xây dựng cấu trúc phần mềm có khả năng mở rộng, đồng thời thiết kế mẫu giao diện thực tế nhằm thử nghiệm hiệu quả của thư viện trên màn hình cảm ứng 7 inch đi kèm ESP32-S3.\\
    \tab Các nội dung chính của đề tài bao gồm:
    \begin{itemize}
        \item \textbf{Nghiên cứu và tích hợp thư viện LVGL:} Làm rõ cách LVGL hoạt động, cách cấu hình và tích hợp với nền tảng ESP-IDF.
        \item \textbf{Xây dựng các thành phần giao diện:} Tạo các thành phần phổ biến như nút, slider, biểu đồ, widget cảm biến... tương thích với hệ thống điều khiển.
        \item \textbf{Thiết kế mô-đun thư viện:} Tổ chức mã nguồn thành các mô-đun có tính tái sử dụng cao, dễ bảo trì và tùy biến theo yêu cầu ứng dụng.
        \item \textbf{Thử nghiệm thực tế:} Ứng dụng thư viện trong một giao diện điều khiển thiết bị đơn giản, hiển thị dữ liệu từ cảm biến nhiệt độ, độ ẩm và trạng thái các thiết bị điều khiển.
    \end{itemize}
    \tab Kết quả cho thấy thư viện hoạt động ổn định, có khả năng phản hồi nhanh và dễ dàng tích hợp vào các dự án thực tế sử dụng ESP32. Đề tài mang lại cái nhìn thực tế về việc xây dựng giao diện trong hệ thống nhúng, góp phần giúp sinh viên nâng cao năng lực lập trình, tư duy hệ thống và ứng dụng công nghệ hiện đại.
    \begin{flushright}
    Thành phố Hồ Chí Minh, tháng 4 năm 2025.
    \end{flushright}
    \end{preface}
    