\begin{preface}{Lời cam đoan}
\tab Chúng em xin cam đoan rằng những nội dung trình bày trong báo cáo luận văn này là công trình nghiên cứu của nhóm dưới sự hướng dẫn của thầy Trần Thanh Bình và thầy Lê Trọng Nhân. Nội dung và số liệu trong báo cáo không phải là bản sao chép từ bất kỳ báo cáo, tiểu luận nào có trước. Tất cả những sự giúp đỡ cho việc xây dựng bài báo cáo đều được trích dẫn và ghi nguồn đầy đủ, rõ ràng. Nếu không đúng sự thật, chúng em xin chịu mọi trách nhiệm trước các thầy, cô, và nhà trường.
\begin{flushright}
Thành phố Hồ Chí Minh, tháng 12 năm 2024.
\end{flushright}
\end{preface}
\newpage
\begin{preface}{Lời cảm ơn}
\tab Chúng em xin được gửi lời cảm ơn chân thành nhất đến thầy Trần Thanh Bình và thầy Lê Trọng Nhân, giảng viên hướng dẫn trực tiếp đề tài. Hai thầy là người đã theo dõi, cũng như góp ý, sửa chữa những sai sót cho chúng em và đã tạo điều kiện làm việc tốt cũng như đã không ngừng hỗ trợ cho nhóm chúng em. Sau một học kỳ thực hiện đề tài, bên cạnh sự nỗ lực của các cá nhân, sự hỗ trợ nhiệt tình từ các thầy và anh đã giúp nhóm em rất nhiều trong việc bắt kịp tiến độ đã đề ra và hoàn thiện hơn đề tài của mình.\\
\tab Chúng em muốn bày tỏ lòng biết ơn sâu sắc đối với những nhận xét và góp ý chân thành của các thầy. Những góp ý của các thầy không chỉ giúp chúng em cải thiện nội dung của luận văn mà còn là nguồn động viên lớn giúp em trở nên tự tin hơn trong việc hoàn thành môn học và bảo vệ đồ án.
Em xin kính chúc Trần Thanh Bình và thầy Lê Trọng Nhân sức khỏe dồi dào và thành công trong công tác giảng dạy. Mong hai thầy hãy tiếp tục mang lại sự tích cực cho các sinh viên mai sau.\\
\tab Cuối cùng, mặc dù đã cố gắng hoàn thành đồ án trong phạm vi và khả năng cho phép, nhưng chúng em chắc chắn không thể tránh khỏi thiếu sót, rất mong nhận được sự góp ý và chỉ bảo của quý thầy cô và các bạn.\\
\begin{flushright}
Chúng em xin chân thành cảm ơn.\\
Thành phố Hồ Chí Minh, tháng 12 năm 2024.\\
\end{flushright}
\end{preface}
\newpage
\begin{preface}{Tóm tắt đồ án}
\tab Đồ án này tập trung vào việc hiện thực một hệ thống cân bằng một loại máy bay không người lái (UAV) được sử dụng rộng rãi trong nhiều ứng dụng từ giám sát đến giao hàng. Mục tiêu của đồ án là phát triển một hệ thống cân bằng cho quadcopter. Trong quá trình nghiên cứu, các thành phần chính của hệ thống đã được xác định gồm:
\begin{itemize}
    \item \textbf{Bộ điều khiển (Controller):} Một thuật toán điều khiển PID (Proportional-Integral-Derivative) đã được áp dụng để điều khiển quadcopter và giữ cho nó ổn định trong không gian.
    \item \textbf{Các cảm biến (Sensors):} Các cảm biến bao gồm mpu6050 và cảm biến áp suất được sử dụng để thu thập dữ liệu về độ cao, các góc xoay và tốc độ của quadcopter.
\end{itemize}
\tab Sau khi hiện thực, hệ thống đã được kiểm tra và đánh giá thông qua một loạt các thử nghiệm mô phỏng và thực tế. Kết quả cho thấy rằng hệ thống đề xuất có khả năng duy trì cân bằng ở một mức độ nhất định, đồng thời cung cấp khả năng điều khiển linh hoạt và ổn định cho quadcopter. Những kết quả này không chỉ mang lại đóng góp cho lĩnh vực nghiên cứu về UAV mà còn có thể được áp dụng trong nhiều ứng dụng thực tế khác, bao gồm giám sát môi trường, giao hàng, và giải cứu khẩn cấp.
\begin{flushright}
Thành phố Hồ Chí Minh, tháng 12 năm 2024.
\end{flushright}
\end{preface}