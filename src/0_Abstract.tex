\begin{preface}{Lời cam đoan}
    \tab Chúng em xin xác nhận rằng báo cáo luận văn này là kết quả của quá trình học tập và nghiên cứu nghiêm túc của nhóm, được thực hiện dưới sự hướng dẫn tận tình của thầy Lê Trọng Nhân. Tất cả nội dung và số liệu trình bày trong báo cáo đều do nhóm tự triển khai, không sao chép từ bất kỳ nguồn tài liệu hay công trình nào trước đó. Những hỗ trợ và tài liệu tham khảo liên quan đều đã được trích dẫn rõ ràng và đầy đủ. Nhóm chúng em xin chịu hoàn toàn trách nhiệm nếu có bất kỳ vi phạm nào liên quan đến tính trung thực và bản quyền của báo cáo.
    \begin{flushright}
    Thành phố Hồ Chí Minh, tháng 4 năm 2025.
    \end{flushright}
    \end{preface}
    \newpage
    \begin{preface}{Lời cảm ơn}
    \tab Chúng em xin gửi lời cảm ơn chân thành và sâu sắc đến thầy Lê Trọng Nhân, giảng viên đã trực tiếp hướng dẫn và đồng hành cùng nhóm trong suốt quá trình thực hiện đề tài. Thầy không chỉ tận tâm theo dõi tiến độ mà còn luôn sẵn sàng đưa ra những góp ý quý báu, kịp thời chỉnh sửa những thiếu sót và tạo điều kiện thuận lợi để nhóm có thể làm việc một cách hiệu quả nhất. Trong suốt một học kỳ, bên cạnh sự nỗ lực của từng thành viên, sự hỗ trợ tận tình của thầy đã giúp nhóm vượt qua nhiều khó khăn, đảm bảo đúng tiến độ và từng bước hoàn thiện đề tài một cách tốt nhất. Chúng em vô cùng trân trọng và biết ơn sự giúp đỡ quý báu ấy.\\
    \tab Chúng em cũng xin bày tỏ lòng biết ơn sâu sắc đến những nhận xét và góp ý chân thành từ thầy. Những lời góp ý không chỉ giúp chúng em nâng cao chất lượng nội dung luận văn, mà còn là nguồn động viên lớn, tiếp thêm sự tự tin để chúng em hoàn thành tốt môn học cũng như phần bảo vệ đồ án.\\
    \tab Chúng em xin kính chúc thầy Lê Trọng Nhân luôn dồi dào sức khỏe, thành công trong công tác giảng dạy và tiếp tục truyền cảm hứng tích cực đến các thế hệ sinh viên sau này.\\
    \tab Cuối cùng, dù đã nỗ lực hết mình để hoàn thành đề tài trong khả năng và thời gian cho phép, nhưng chắc chắn vẫn còn những thiếu sót. Chúng em rất mong nhận được sự góp ý và chỉ dẫn thêm từ thầy và quý thầy cô để hoàn thiện hơn trong tương lai.
    \begin{flushright}
    Chúng em xin chân thành cảm ơn.\\
    Thành phố Hồ Chí Minh, tháng 4 năm 2025.\\
    \end{flushright}
    \end{preface}
    \newpage
    \begin{preface}{Tóm tắt đồ án}
    \tab Đồ án này tập trung vào việc xây dựng một hệ thống nhà thông minh, sử dụng vi điều khiển ESP32-S3 Touch 7 inch LCD làm trung tâm điều khiển, kết hợp với phần mềm thiết kế giao diện người dùng SquareLine Studio và nền tảng CoreIOT để giám sát và điều khiển thiết bị trong nhà theo thời gian thực. Trong quá trình nghiên cứu, các thành phần chính của hệ thống đã được xác định gồm:
    \begin{itemize}
        \item \textbf{Giao diện điều khiển cảm ứng:} Thiết kế trên SquareLine, chạy trực tiếp trên màn hình 7 inch của ESP32-S3, cho phép người dùng tương tác trực quan để bật/tắt các thiết bị điện, theo dõi trạng thái môi trường và điều khiển theo từng khu vực.
        \item \textbf{Mạch relay điều khiển thiết bị:} Được điều khiển bởi ESP32 để quản lý các thiết bị đầu ra như đèn, quạt, máy bơm,… thông qua giao diện cảm ứng hoặc từ xa qua internet.
        \item \textbf{Cảm biến nhiệt độ và độ ẩm sử dụng giao tiếp RS485:} Hệ thống sử dụng các cảm biến RS485 để thu thập dữ liệu nhiệt độ và độ ẩm từ nhiều khu vực trong ngôi nhà. Các dữ liệu môi trường này được gửi về ESP32 để hiển thị lên giao diện người dùng và đồng bộ lên CoreIOT.
        \item \textbf{Nền tảng IoT – CoreIOT:} Hỗ trợ kết nối với hệ thống mạng, cho phép người dùng giám sát trạng thái và điều khiển thiết bị từ xa, đồng thời ghi nhận lịch sử dữ liệu môi trường và thiết bị.
    \end{itemize}
    \tab Sau quá trình thiết kế, lập trình và kiểm thử, hệ thống đã cho thấy hiệu quả hoạt động ổn định, khả năng phản hồi nhanh và giao diện dễ sử dụng. Việc tích hợp cảm biến RS485 cho phép mở rộng quy mô và tăng độ chính xác trong việc giám sát môi trường. Đồ án không chỉ giúp sinh viên làm quen với các công nghệ hiện đại trong lĩnh vực IoT, nhúng và giao tiếp công nghiệp, mà còn mở ra hướng ứng dụng thực tiễn trong các mô hình nhà ở thông minh.
    \begin{flushright}
    Thành phố Hồ Chí Minh, tháng 4 năm 2025.
    \end{flushright}
    \end{preface}